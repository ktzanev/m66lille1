\documentclass[a4paper,12pt,reqno]{amsart}
\usepackage{macros_M66}

\begin{document}
%%%%%%%%%%%%%%%%%%%%%%%%%%%%%%%%%%%%%%%%%%%%%%%%%
\hautdepage{TP4: Équations Différentielles Ordinaires}
%%%%%%%%%%%%%%%%%%%%%%%%%%%%%%%%%%%%%%%%%%%%%%%%%

Vous êtes invités à télécharger le fichier \cmd{tp4\_fonction.sci} à partir de moodle, puis à le compléter au fur et à mesure avec les nouvelles procédures. Pour chaque exercice il va falloir créer un fichier, \cmd{tp4\_exo1.sce}, \dots, \cmd{tp4\_exo4.sce}, comportant les lignes de code correspondant à la résolution de l'exercice et incluant au début le fichier \cmd{tp4\_fonction.sci} et les autres initialisations habituelles (\sclb{clear},\sclb{clc},\ldots).

% ==================================
\section{Rappel méthodes à un pas}
% ==================================

L'objectif des méthodes numériques pour les EDO est de calculer une valeur approchée de la solution du problème de Cauchy :
\begin{equation}\label{pbcauchy}\tag{$\star$}
    \begin{cases}
      u'(t)=f(t,u(t)),\quad t>0 \\
      u(0)=u_0
    \end{cases}
\end{equation}
avec $f: \mathbb{R} \times \mathbb{R}^d \to  \mathbb{R}^d$, $d \geq 1$.

\attention{center}{Dans ce TP, pour simplifier, on se limitera au cas autonome, $f$ ne dépendra pas de $t$, et aux méthodes à pas constant.}

Afin de calculer une solution approchée de \eqref{pbcauchy} sur un intervalle $[0,T]$, on se donne une subdivision régulière de $[0,T]$ :
$(t_n)_{0\leq n\leq N}$ avec $t_n=nh$ où $h=T/N$ est le pas de la subdivision.
On veut calculer une suite $(u_n)_{0\leq n\leq N}$ telle
que $u_n$ soit une \og{}bonne\fg{} approximation de $u(t_n)$ ($u$ solution
exacte). \\
Les méthodes utilisées dans ce TP sont :

\subsection*{Méthode d'Euler explicite:} (à programmer)\newline
$ u_{n+1}=u_n+h f(u_n)$, avec $u_0$ pour condition initiale.

\subsection*{Méthode d'Euler implicite:} (disponible dans \cmd{tp4\_fonction.sci})\newline
$ u_{n+1}=u_n+h f(u_{n+1})$, avec $u_0$ pour condition initiale.

\subsection*{Méthode de Crank-Nicolson:} (disponible dans \cmd{tp4\_fonction.sci})\newline
\smallskip
\centerline{$
u_{n+1}=u_n+\frac{h}{2}  (f(u_n)+f(u_{n+1}))$, avec $u_0$ pour condition initiale.}

\subsection*{Méthodes de Runge-Kutta (RK2):}  (à programmer)\newline
Euler amélioré (avec $\alpha = \frac{1}{2}$) et Heun (avec $\alpha=1$) :
$$
  \begin{cases}
    k_{n,1}=f(u_n)\\
    k_{n,2}=f(u_n+\alpha h k_{n,1})\\
    u_{n+1}=u_n+h \bigl( \bigl(1 - \frac{1}{2\alpha}\bigl) k_{n,1}+\frac{1}{2\alpha}k_{n,2}\bigl),
  \end{cases}
$$
avec $u_0$ pour condition initiale.

\subsection*{Méthode de Runge-Kutta (RK4):} (disponible dans \cmd{tp4\_fonction.sci})\newline
$$
  \begin{cases}
    k_{n,1}=f(u_n)\\
    k_{n,2}=f(u_n+\frac{h}{2}k_{n,1})\\[0.2cm]
    k_{n,3}=f(u_n+\frac{h}{2} k_{n,2})\\[0.2cm]
    k_{n,4}=f(u_n+h k_{n,3})\\[0.2cm]
    u_{n+1}=u_n+\frac{h}{6}\bigl( k_{n,1}+2 k_{n,2}+2 k_{n,3}+k_{n,4}\bigl),
  \end{cases}
$$

avec $u_0$ pour condition initiale.

% ==================================
\section{La méthode d'Euler explicite }
% ==================================


% --------------------------------------------
\begin{exo} (La méthode d'Euler explicite)
% --------------------------------------------
  \begin{enumerate}\label{ex:euler}

    \item Programmer la méthode d'Euler explicite \sclb{[U]=EulerExplicite(f,T,N,U0)} qui étant donnés une fonction \sclb{f}, la borne supérieure \sclb{T} de l'intervalle $[0,T]$, le nombre de subdivisions \sclb{N} de cette intervalle et la condition initiale \sclb{U0} (qui est un vecteur colonne), retourne une matrice \sclb{U} dont les $N+1$ colonnes représentent les valeurs approchées (en particulier \sclb{U[1]} = \sclb{U0}).

    Cette méthode est à mettre dans \cmd{tp4\_fonction.sci} et sera utilisée à plusieurs reprises dans ce TP.

    \item On considère le problème de Cauchy suivant :
      $$
        \begin{cases}
          y'(t)=-10y(t), \quad t \in [0,1], \\
          y(0)=2.
        \end{cases}
      $$
    Définissez une fonction \sclb{solex} qui est la solution exacte de ce problème.

    \item Tracer sur une même figure la solution exacte et les solutions approchées obtenues par la méthode d'Euler explicite pour $N=2^{k}$ ($h=2^{-k}$) avec $k=4,\dots,10$.

    \item\label{q:oceuler} Comme la solution exacte $y$ du problème est ${\mathcal{C}}^{2}$, on rappelle que la méthode d'Euler explicite vérifie :
    $$
      \max_{0 \leq n \leq N} |y_n-y(t_n)|\leq C h \sup_{0 \leq t \leq T} |y^{(2)}(t)|.
    $$
    Dans une autre figure, tracer la norme infinie de l'erreur commise en échelle \og{}log-log\fg{} en fonction du pas $h=T/N$, pour les mêmes $N=2^{k}$ ($h=2^{-k}$) avec $k=4,\dots,10$, ainsi qu'une droite de pente 1.
    Que constatez-vous?
    \item
    On considère les deux problèmes de Cauchy suivants :
    $$
      \begin{cases}
        y'(t)=3y(t), \quad t \in [0,20], \\
        y(0)=1/3.
      \end{cases}
      , \quad
      \begin{cases}
        y'(t)=-3y(t), \quad t \in [0,20], \\
        y(0)=1/3.
      \end{cases}
    $$
    Quelle sont les solutions exactes de ces deux problèmes ? Mettre en évidence  pour chacune d'entre elles la sensibilité de la solution numérique par rapport à une petite perturbation aléatoire dans la condition initiale. Pour cela, effectuer une résolution par la méthode d'Euler explicite avec $y(0)= 1/3 + \varepsilon$, où $\varepsilon \leq 10^{-7}$ est une perturbation aléatoire. Représenter les résultats et les commenter.
  \end{enumerate}
\end{exo}

% ==================================
\section{Comparaison  de quelques schémas numériques}
% ==================================


% --------------------------------------------
\begin{exo} (Ordres de convergence)
% --------------------------------------------

  On se propose dans cet exercice de comparer les ordres de convergence des différentes méthodes décrites dans l'introduction de cette feuille.

  On rappelle que, en supposant que les solutions $y$ de l'équation différentielle soient de classe ${\mathcal{C}}^{p+1}$, une méthode numérique est d'ordre de convergence $p$ si :
    $$
      \max_{0 \leq n \leq N} |y_n-y(t_n)|\leq C h^p \sup_{0 \leq t \leq T} |y^{(p+1)}(t)|.
    $$
  Trois des méthodes dont on aura besoin, Euler implicite, Crank-Nicolson et Runge-Kutta 4, sont fournies dans le fichier \cmd{tp4\_fonction.sci}. Les deux méthodes implicites exige comme paramètre, en plus de la fonction \sclb{f}, sa dérivée \sclb{Df} pour pouvoir résoudre l'équation de récurrence grâce à la méthode de Newton.
  \begin{enumerate}
    \item
    En plus de la méthode d'Euler explicite déjà implémentée, programmer les méthodes d'Euler amélioré \sclb{[U]=EulerAmeliore(f,T,N,U0)} et de Heun \sclb{[U]=Heun(f,T,N,U0)}.
    \item On considère l'équation différentielle linéaire :
      $$
        \begin{cases}
          y'(t) = -y(t), \; t \in [0,5],\\
          y(0) = 2.
        \end{cases}
      $$
    Comme dans \ref{ex:euler}.\ref{q:oceuler}), on se propose de résoudre puis d'évaluer les erreurs commises en norme infinie pour toutes les méthodes, les trois que vous avez programmées et les trois fournies.
    Pour cela, faire les calculs pour $N=2^{k}$ ($h=\sfrac{5}{2^{k}}$) avec $k=4,\dots,10$. Puis mettre en évidence  les ordres de convergence en traçant, sur un même graphique, la norme infinie de l'erreur commise en échelle \og{}log-log\fg{} pour chacune de ces méthodes, ainsi que 3 droites respectivement de pente $1$, $2$ et $4$.\newline
    Que constatez-vous?
  \end{enumerate}
\end{exo}



% ==================================
\section{Systèmes différentiels à coefficients constants}
% ==================================


% --------------------------------------------
\begin{exo} (Un problème raide)
% --------------------------------------------

On considère le système :
  $$
    \begin{cases}
      y'_1(t) = y_2(t),\\
      y'_2(t) = -\lambda_1 \, \lambda_2 y_1(t)+(\lambda_1+\lambda_2) y_2(t),
    \end{cases}
  $$
avec $\lambda_1$ et $\lambda_2$ deux réels négatifs distincts tels que $|\lambda_1| >> |\lambda_2|$. On  cherche à calculer la solution sur $[0,20]$.

On fixe pour cet exercice $\lambda_1=-100$, $\lambda_2=-1$ et la condition initiale $y_{1}(0)=y_{2}(0)=1$.
\begin{enumerate}
  \item En utilisant la fonction \sclb{expm}, donner la solution exacte du problème \sclb{y=solex(t)}.
  \item Résoudre le problème par les méthodes d'Euler explicite et implicite pour les nombres de subdivisions $N = 966$ ($h\sim 0.0207$), $N = 1030$ ($h\sim 0.0194$) et $N = 2000$ ($h = 0.01$). Pour chacune des trois valeurs de $N$, tracer sur un graphique les deux composantes de $Y_{n}$, et sur un autre graphique la trajectoire de $Y_{n}$.
  Que constatez-vous?

  \textit{Indication : Pour les méthodes itératives sur $\mathbb{R}^{n}$, la \og{}dérivée\fg{} \sclb{Df} est la matrice jacobienne de \sclb{f}.}
\end{enumerate}
\end{exo}


% --------------------------------------------
\begin{exo} (Modèle de Lotka-Volterra)
% --------------------------------------------

  On veut appliquer les méthodes d'Euler explicite, d'Euler implicite, d'Euler amélioré, de Heun, de Crank-Nicolson et de Runge-Kutta 4
  à la résolution du modèle de Lotka-Volterra :
    $$
      \begin{cases}
        x' =ax-bxy\\
        y' =-cy+dxy
      \end{cases}
    $$
  Pour cet exercice, on fixe les constantes et la condition initiale comme suit :
    $$
    a=3,\ b=1,\ c=2,\ d=1,\ x(0)=1,\ y(0)=2.
    $$
  \begin{enumerate}
    \item Calculer la solution sur l'intervalle $[0,T]$, \sclb{T=10}, avec \sclb{N = 100} subdivisions. Représenter la solution numérique de chaque méthode dans le plan de phase $(x,y)$ (marquer le point initial). Commenter les résultats.
    \item On veut comparer les méthodes d'Euler amélioré, de Heun, de Crank-Nicolson et de Runge-Kutta 4 sur un intervalle plus long $[0,T]$, \sclb{T=400}. Pour faire cela, tracer sur un même graphique l'évolution en temps de l'intégrale première
      $$
        H(x,y) = dx -c \ln x + by -a \ln y,
      $$
    avec un pas de discrétisation constant deux fois plus grand que dans la question précédente. Discuter la stabilité de ces méthodes quand le temps $T$ devient grand.

    \textit{Rappel : L'intégrale première $H(x,y)$ est constante le long des solutions du système.}
  \end{enumerate}
\end{exo}
\end{document}
