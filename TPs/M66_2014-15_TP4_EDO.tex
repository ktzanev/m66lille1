\documentclass[a4paper,12pt,reqno]{amsart}
\usepackage{macros_M66}

\begin{document}
%%%%%%%%%%%%%%%%%%%%%%%%%%%%%%%%%%%%%%%%%%%%%%%%%
\hautdepage{TP4: Equations Diff\'erentielles Ordinaires}
%%%%%%%%%%%%%%%%%%%%%%%%%%%%%%%%%%%%%%%%%%%%%%%%%

\section{Rappel m\'ethodes \`a un pas}
%%%%%%%%%%%%%%%%%%%%%%%%%%%

L'objectif des m\'ethodes num\'eriques pour les EDO est de calculer une valeur approch\'ee de la solution du probl\`eme de Cauchy :
\begin{equation}\label{pbcauchy}
\left\{\begin{array}{l}
u'(t)=f(t,u(t)),\ t>0\\
u(0)=u_0
\end{array}
\right.
\end{equation}
avec $f:\R\times\R^d\to \R^d, d\geq 1$.
Dans la suite, on se limitera au cas autonome : $f$ ne d\'ependra pas de $t$.

Afin de calculer une solution approch\'ee de \eqref{pbcauchy} sur un intervalle $[0,T]$, on se donne une
subdivision de $[0,T]$, r\'eguli\`ere pour simplifier :
$(t_n)_{0\leq n\leq N}$ avec $t_n=nh$ o\`u $h=T/N$ est le pas de la subdivision.  
On veut calculer une suite $(u_n)_{0\leq n\leq N}$ telle
que $u_n$ soit une ``bonne'' approximation de $u(t_n)$ ($u$ solution
exacte). \\

{\bf M\'ethode d'Euler explicite :}
$ u_{n+1}=u_n+h f(t_n,u_n)$, avec $u_0$ pour condition initiale.

{\bf M\'ethode d'Euler implicite :}
$ u_{n+1}=u_n+h f(t_{n+1},u_{n+1})$, avec $u_0$ pour condition initiale.

{\bf M\'ethode de Crank-Nicolson :}
\smallskip
\centerline{$
u_{n+1}=u_n+\frac{h}{2}  (f(t_n,u_n)+f(t_{n+1},u_{n+1}))$, avec $u_0$ pour condition initiale.} 

{\bf M\'ethodes de Runge-Kutta (RK2)} : Euler am\'elior\'e (avec $\alpha = \frac{1}{2}$) et Heun (avec $\alpha=1$) :
\smallskip
$$\left\{
\begin{array}{l}
k_{n,1}=f(t_n,u_n)\\
k_{n,2}=f(t_n+ \alpha h,u_n+\alpha h k_{n,1})\\
u_{n+1}=u_n+h \bigl( \bigl(1 - \frac{1}{2\alpha}\bigl) k_{n,1}+\frac{1}{2\alpha}k_{n,2}\bigl),
\end{array}\right.
$$
avec $u_0$ pour condition initiale.

{\bf M\'ethode de Runge-Kutta (RK4) :}
\smallskip
$$\left\{
\begin{array}{l}
k_{n,1}=f(t_n,u_n)\\
k_{n,2}=f(t_{n+1/2},u_n+\frac{h}{2}k_{n,1})\\[0.2cm]
k_{n,3}=f(t_{n+1/2},u_n+\frac{h}{2} k_{n,2})\\[0.2cm]
k_{n,4}=f(t_{n+1},u_n+h k_{n,3})\\[0.2cm]
u_{n+1}=u_n+\frac{h}{6}\bigl( k_{n,1}+2 k_{n,2}+2 k_{n,3}+k_{n,4}\bigl),
\end{array}\right.
$$
avec $u_0$ pour condition initiale.

Pour les m\'ethodes implicites, il faut aussi programmer la m\'ethode de Newton pour r\'esoudre l'\'equation 
$\Phi (Y) = 0$ (\'equation non lin\'eaire scalaire ou vectorielle).

%%%%%%%%%%%%%%%%%%%%%%%%%%%%%
\section{La m\'ethode d'Euler explicite }
%%%%%%%%%%%%%%%%%%%%%%%%%%%%%

\begin{exo} La m\'ethode d'Euler explicite : le cas scalaire
%%%%%%%%%%%%%%%%%%%%%%%%%%%
\begin{enumerate}
\item
On consid\`ere le probl\`eme de Cauchy suivant :
$$
\left\{
\begin{array}{lcl}
y'(t)=-10y(t), \quad t \in [0,1], \\
y(0)=2.
\end{array}
\right.
$$
En programmant la m\'ethode d'Euler explicite pour la r\'esolution de ce probl\`eme, mettre en \'evidence l'ordre de convergence de la m\'ethode, en prenant le pas de discr\'etisation uniforme $h=1/2^k$, $k=4,\dots,10$. 
En supposant $y \in {\mathcal{C}}^{2}$, on rappelle que la m\'ethode d'Euler explicite v\'erifie :
$$
\max_{0 \leq n \leq N} |y_n-y(t_n)|\leq C h \sup_{0 \leq t \leq T} |y^{(2)}(t)|.
$$
Tracer dans une figure la solution exacte avec pas $h=2^{-10}$ et la solution approch\'ee avec diff\'erents pas $h$.
Dans une autre figure, tracer la norme infinie de l'erreur commise en \'echelle "log-log" ainsi qu'une droite de pente 1. 
Que constatez-vous?
\item
On consid\`ere les deux probl\`emes de Cauchy suivants :
$$
\left\{
\begin{array}{lcl}
y'(t)=3y(t)-3t, \quad t \in [0,20], \\
y(0)=1/3.
\end{array}
\right.
, \quad 
\left\{
\begin{array}{lcl}
y'(t)=-3y(t)+3t, \quad t \in [0,20], \\
y(0)=1/3.
\end{array}
\right.
$$
Comparer les solutions exactes de ces deux probl\`emes, puis mettre en \'evidence  pour chacun d'entre eux la sensibilit\'e de la solution num\'erique par rapport \`a une petite perturbation dans la condition initiale en effectuant une r\'esolution par la m\'ethode d'Euler explicite (consid\'erer $y(0)= 1/3 + \epsilon$, avec $\epsilon \approx 10^{-15}$ et le pas de discr\'etisation uniforme $h=0.01$).
\end{enumerate}
\end{exo}

%%%%%%%%%%%%%%%%%%%%%%%%%%%%%
\section{Comparaison  de quelques sch\'emas num\'eriques}
%%%%%%%%%%%%%%%%%%%%%%%%%%%%%

\begin{exo} Ordres de convergence
%%%%%%%%%%%%%%%%%%

On consid\`ere l'\'equation diff\'erentielle lin\'eaire :
$$
\left\{
\begin{array}{rcl}
y'(t)&=&-y(t), \; t \in [0,5],\\
y(0)&=&2.
\end{array}
\right.
$$
On d\'esire mettre en oeuvre quelques m\'ethodes num\'eriques pour l'approximation de la solution de cette \'equation diff\'erentielle. 
On se donne $h$ le pas de la subdivision uniforme $(t_n)_{n=0,..N}$ de l'intervalle $[0,5]$, on note $t_n=n\, h$ et $y_n$ une approximation de $y(t_n)$. On rappelle, en supposant $y \in {\mathcal{C}}^{p+1}$, qu'une m\'ethode num\'erique est d'ordre de convergence $p$ si :
$$
\max_{0 \leq n \leq N} |y_n-y(t_n)|\leq C h^p \sup_{0 \leq t \leq T} |y^{(p+1)}(t)|.
$$
\begin{enumerate}
\item En plus de la m\'ethode d'Euler explicite d\'ej\`a impl\'ement\'ee, programmer les m\'ethodes d'Euler am\'elior\'e et de Heun.
\item Consid\'erer les m\'ethodes d'Euler implicite, de Crank-Nicolson et de Runge-Kutta 4, fournies dans le fichier {\tt tp\_EDO\_functions.sci} pour la r\'esolution de cette EDO en prenant le pas de temps $h=1/2^k$, $k=4,\dots,10$. 
Mettre en \'evidence  les ordres de convergence de chacune de ces m\'ethodes en tra\c cant la norme infinie de l'erreur commise en \'echelle "log-log" pour chacune de ces m\'ethodes ainsi que 3 droites respectivement de pente 1, 2 et 4. 
Que constatez-vous?
\end{enumerate}
\end{exo}

\begin{exo} Et si on explose en temps fini ?
%%%%%%%%%%%%%%%%%%%%%%%

On consid\`ere l'\'equation diff\'erentielle :
$$
\left\{
\begin{array}{lcl}
u'(t)&=&u^2(t), \\
u(0)&=&1.
\end{array}
\right.
$$
\begin{enumerate}
\item Donner la solution maximale de cette \'equation diff\'erentielle.
\item
Donner explicitement la valeur de $u_{n+1}$ en fonction de $u_n$ lorsque cette \'equation 
diff\'erentielle est r\'esolue avec le sch\'ema d'Euler implicite. Quelle restriction doit-on observer sur le pas de temps $h$ ?  
Proposer alors une strat\'egie mettant en oeuvre un pas de temps variable, qui d\'etecte l'explosion de la solution et s'arr\^ete \`a temps avant l'obtention d'une erreur d'execution.
\item Donner explicitement la valeur de $u_{n+1}$ en fonction de $u_n$ lorsque cette \'equation 
diff\'erentielle est r\'esolue avec le sch\'ema de Crank-Nicolson. Quelle restriction doit-on observer sur le pas de temps $h$ ?  
Proposer alors une strat\'egie mettant en oeuvre un pas de temps variable, qui d\'etecte l'explosion de la solution et s'arr\^ete \`a temps avant l'obtention d'une erreur d'execution.
\item Programmer le sch\'ema d'Euler explicite avec un pas de temps variable. Soit $h_0=0.01$ et $\mu=10^{-4}$.
On fixe un encadrement $[h_{min},h_{max}]=[0.1 h_0, 5 h_0]$ et on choisit $h_n \in [h_{min},h_{max}]$ selon le crit\`ere :
\begin{enumerate}
\item si $\frac \mu 3 \leq \frac{|e^*_n|}{h_n} \leq \mu$, alors $h_{n+1}=h_n$ ;
\item si $\frac{|e^*_n|}{h_n} < \frac \mu 3$, alors $h_{n+1}=\min(1.1 h_n,h_{max})$ ;
\item si $\frac{|e^*_n|}{h_n} > \mu$, alors $h_{n+1}=\max(0.9 h_n,h_{min})$.
\end{enumerate}
Pour calculer $e^*_n$, approximation de l'erreur \`a l'\'etape $n$, il faut :
\begin{enumerate}
\item \'evaluer $p_n=f(t_n,y_n)$ et $p_{n+1}=f(t_n+h_n,y_n+h_n p_n)$ ;
\item calculer l'approximation de l'erreur $\frac{e^*_n}{h_n}= \frac{(p_{n+1}-p_n)}{2}$. 
\end{enumerate}
On remarque que ceci ne n\'ecessite aucun calcul suppl\'ementaire car $p_{n+1}$ est de toute fa\c con n\'ecessaire pour l'\'etape suivante.
\item
Comparer les diff\'erents sch\'emas. Quels sont les avantages et inconv\'enients de chacun d'entre eux ? 
D'apr\`es les erreurs num\'eriques observ\'ees, quel vous semble le sch\'ema le plus adapt\'e pour la r\'esolution de cette \'equation diff\'erentielle ?
\end{enumerate}
\end{exo}

%%%%%%%%%%%%%%%%%%%%%%%%%%%%%
\section{Syst\`emes diff\'erentiels \`a coefficients constants}
%%%%%%%%%%%%%%%%%%%%%%%%%%%%%

\begin{exo} Les probl\`emes raides
%%%%%%%%%%%%%%%%%%

On consid\`ere le syst\`eme :
$$
\left\{
\begin{array}{lcl}
y'_1(t)&=&y_2(t),\\
y'_2(t)&=&-\lambda_1 \, \lambda_2 y_1(t)+(\lambda_1+\lambda_2) y_2(t),
\end{array}
\right.
$$
avec $y_1(0)=y_{1,0}$, $y_2(0)=y_{2,0}$, $\lambda_1$ et $\lambda_2$ deux r\'eels n\'egatifs distincts tels que $|\lambda_1| >> |\lambda_2|$. On  cherche \`a calculer la solution sur $]0,20[$.
\begin{enumerate}
\item Donner la solution analytique du probl\`eme.
\item En prenant $\lambda_1=-100$, $\lambda_2=-1$, $y_{1,0}=y_{2,0}=1$, r\'esoudre le probl\`eme par la m\'ethode d'Euler explicite en prenant d'abord $h=0.0207$ puis $h=0.0194$ et $h=0.01$. Que constatez-vous? 
R\'esoudre le m\^eme probl\`eme par la m\'ethode d'Euler implicite en prenant $h=0.1$. Que constatez-vous?
\end{enumerate}
\end{exo}

\begin{exo} Mod\`ele de Lotka-Volterra
%%%%%%%%%%%%%%%%%%%%%

On veut appliquer les m\'ethodes d'Euler explicite, d'Euler implicite, d'Euler am\'elior\'e, de Heun, de Crank-Nicolson et de Runge-Kutta 4
\`a la r\'esolution du mod\`ele de Lotka-Volterra :
 $$
 \left\{\begin{aligned}
x' \ =ax-bxy,\\
y'=-cy+dxy,
 \end{aligned}
 \right.
 \quad\mbox{ avec } x(0)=x_0,\ y(0)=y_0.
 $$
 Pour la simulation num\'erique, on pourra prendre :
 $$
 a=3,\ b=1,\ c=2,\ d=1,\ x_0=1,\ y_0=2.
 $$
\begin{enumerate}
\item Calculer la solution sur l'intervalle $[0,T], T=10$ avec un pas de discr\'etisation constant $h= 0.05$.
Repr\'esenter la solution num\'erique de chaque m\'ethode dans le plan de phase $(x,y)$ (rep\'erer par un $\oplus$ la condition initiale).
Commenter les r\'esultats.
\item On veut comparer les m\'ethodes d'Euler am\'elior\'e, de Heun, de Crank-Nicolson et de Runge-Kutta 4 sur l'intervalle $[0,T], T=100$. Pour faire cela, 
tra\c cer l'\'evolution en temps de l'int\'egrale premi\`ere $H(x,y) = dx -c \ln x + by -a \ln y$, avec un pas de discr\'etisation 
constant $h= 0.05$. Discuter de la stabilit\'e de ces m\'ethodes quand le temps $T$ devient grand.
\end{enumerate}
\end{exo}

\end{document}
