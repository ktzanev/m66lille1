\documentclass[a4paper,12pt,reqno]{amsart}
\usepackage{macros_M66}

\begin{document}
%%%%%%%%%%%%%%%%%%%%%%%%%%%%%%%%%%%%%%%%%%%%%%%%%
\hautdepage{TP1: Interpolation polynomiale}
%%%%%%%%%%%%%%%%%%%%%%%%%%%%%%%%%%%%%%%%%%%%%%%%%

Vous êtes invités à télécharger le fichier \cmd{tp1\_fonction.sci} à partir de moodle, puis à le compléter au fur et à mesure avec les nouvelles procédures. Pour chaque exercice il va falloir créer un fichier, \cmd{tp1\_exo1.sce}, \cmd{tp1\_exo2.sce} et \cmd{tp1\_exo3.sce}, comportant les lignes de code correspondant à la résolution de l'exercice et incluant au début le fichier \cmd{tp1\_fonction.sci} et les autres initialisations habituelles (\sclb{clear},\sclb{clc},\ldots).

\attention{center}{En cas de blocage, commencez toujours par regarder l'aide ou sur Google !!}


% ==================================
\section{Présentation et indications}
% ==================================

Soient $n+1$ points distincts $(x_i)_{0 \leq i \leq n}$ de $[a,b] \subset \R$, et $f$ une fonction réelle définie sur $[a,b]$. Nous cherchons à construire le polynôme d'interpolation de Lagrange $p_n$ de degré au plus égal à $n$ tel que $p_n(x_i)=f(x_i)$ pour $ 0 \leq i \leq n$. On exprime ce polynôme dans sa base de Newton, sous la forme:
    $$
        p_n(x)= f[x_0]+ \sum_{i=1}^n f[x_0,\ldots,x_i] \prod_{k=0}^{i-1} (x-x_k).
    $$

L'année dernière, en L2, vous avez déjà travaillé sur ce problème en Scilab. Pour ne pas devoir refaire le même travail, on vous fournit dans le fichier \cmd{tp1\_fonction.sci} deux fonctions :
\begin{enumerate}
     \item \sclb{function [cfs]=DiffDiv(xdata,ydata)} qui à partir des points $\{x_0,\ldots,x_n\}$ contenus dans \sclb{xdata} et des valeurs de la fonctions $\{f(x_0),\ldots,f(x_n)\}$ contenus dans \sclb{ydata}, retourne les différences divisée \sclb{cfs}$=\{f[x_0],\ldots,f[x_0,\ldots,x_n]\}$.
     \item \sclb{function [y]=HornerNewton(cfs,xdata,t)} qui à partir des différences divisée \sclb{cfs} et les points \sclb{xdata}, retourne la valeur du polynôme d'interpolation évalué au point \sclb{t}.
 \end{enumerate}


% ==================================
\section{Exercices}
% ==================================

% --------------------------------------------
\begin{exo} (Vérification de l'interpolation)

  \begin{enumerate}
    \item Commencer par crée le fichier \cmd{tp1\_exo1.sce} qui sera initialisé par les commandes suivantes : \sclb{clc}, \sclb{clear}, \sclb{xdel(winsid())}, \sclb{exec('tp1_fonctions.sci')}. Commenter ces commandes.

    \item Créer dans ce même fichier la fonction polynomiale $f_{1}(x)=x^{2}+2x$, nommée \sclb{f1}. Puis vérifier qu'elle fonctionne.

    \item Dans le fichier \cmd{tp1\_fonction.sci} créer la procédure
    \begin{center}
      \sclb{function PlotFunction(f,a,prec)}
    \end{center}
    qui trace la fonction \sclb{f} sur l'intervalle \sclb{[-a,a]} avec un pas de précision \sclb{prec}. Tester cette procédure (dans \cmd{tp1\_exo1.sce}) avec la fonction $f_{1}$.

    \item Dans le fichier \cmd{tp1\_fonction.sci} créer la procédure
    \begin{center}
      \sclb{function PlotInterpUnif(f,a,k,prec)}
    \end{center}
    qui trace, avec un pas de précision \sclb{prec}, le polynôme d'interpolation uniforme\footnote{c'est-à-dire, dont les points d'interpolations sont équidistants.} de degré au plus \sclb{k} de \sclb{f} sur l'intervalle \sclb{[-a,a]}.

    \item Sur la même figure tracer la fonctions $f_{1}$ sur $[-3,3]$ et ces polynômes d'interpolation uniforme $p_{k}$ de degré majoré par $k=0,..,10$. Que constatez-vous ?

    \item Écrire un test qui imprime le degré du polynôme $p_{10}$. Pour déterminer le degré on peut ce servir des fonctions \sclb{find}, \sclb{max}, et pour l'impression de \sclb{disp} et \sclb{sprintf}.\newline
    Le résultat est-il cohérent avec la théorie? Explication.

    \item En utilisant la fonction \sclb{norm} calculer et afficher les erreurs d'interpolation $\|f_{1}-p_{2}\|_{\infty}$ et $\|f_{1}-p_{10}\|_{\infty}$. Commenter.

  \end{enumerate}
\end{exo}


% --------------------------------------------
\begin{exo} (Phénomène de Runge)

  Soit $f \in {\mathcal{C}}^{n+1}([a,b])$. On rappelle le résultat :
  $$
    \forall x \in [a,b], \exists \; \zeta \in [a,b] \; ; \;  f(x)-p_n(x)=\frac{f^{(n+1)}(\zeta)}{(n+1)!} \prod_{i=0}^n (x-x_i).
  $$

  \begin{enumerate}
    \item Dans le fichier \cmd{tp1\_fonction.sci} créer la procédure
    \begin{center}
      \sclb{function PlotErrInterpUnif(f,a,deg,prec)}
    \end{center}
    qui, après avoir déterminé le polynôme d'interpolation uniforme $p_k$, calcule l'erreur d'interpolation $\| f - p_{k} \|_{\infty}$ sur \sclb{[-a,a]} en utilisant la fonction \sclb{norm} avec un pas de précision \sclb{prec}, pour les degrés $k$ prescrits dans le vecteurs \sclb{deg}. Puis tracer ces erreurs (en échelle $\log$) en fonctions du degré.\newline
    Par exemple la commande \sclb{PlotErrInterpUnif(f,3,[10:20],.1)} doit calculer et tracer les erreurs d'approximation (calculées avec un pas de précision \sclb{.1}) de \sclb{f} sur \sclb{[-3,3]} pour les polynômes $p_{10},\ldots,p_{20}$.

    \item Créer la fonction $f_{2}(x)=\sin(\pi x)$, appelé \sclb{f2}, dans le fichier \cmd{tp1\_exo2.sce}. N'oublier pas d'initialiser ce fichier au début, comme vous l'avez fait pour \cmd{tp1\_exo1.sce}.

    \item Tracer cette fonction sur $[-2,2]$ ainsi que ses polynômes interpolations pour les degrés $k=20,\ldots,30$ sur le même graphique.

    \item Tracer l'erreur d'interpolation uniforme de $f_{2}$ pour les degrés $[0,1,\ldots,50]$. Est-ce prévisible ? Quelle est le résultat théorique qu'on peut annoncer.

    \item Soit $f_{3}=\frac{1}{1+x^{2}}$. Refaire les même calculs que pour $f_{2}$.

    \item Puis élargir l'intervalle de $[-2,2]$ à $[-5,5]$. Que constate t-on ? Comment l'expliquer ?
  \end{enumerate}
\end{exo}


% --------------------------------------------
\begin{exo} (Abscisses de Tchebychev)

  Étant donné un entier $n \in \N^*$ et un intervalle $[a,b]$, on rappelle que les $n$ abscisses de Tchebychev dans $[a,b]$ sont donnés par:
  $$
    x_i=\frac{a+b}{2}+\frac{b-a}{2} \hat{x}_i, \mbox{ où } \hat{x}_i=\cos\left(\displaystyle \frac{(2i+1)}{2n} \pi \right), \quad i=0,...,n-1.
  $$
  \begin{enumerate}

    \item Dans le fichier \cmd{tp1\_fonction.sci} créer la procédure
    \begin{center}
      \sclb{function [v]=tcheb(a,b,n)}
    \end{center}
    qui renvoie les \sclb{n} abscisses de Tchebychev dans \sclb{[a,b]}.

    \item Modifier (copier/coller + renommer + modifier) les fonctions\\
    \sclb{PlotInterpUnif} $\longrightarrow$ \sclb{PlotInterpTcheb}\\
    \sclb{PlotErrInterpUnif} $\longrightarrow$ \sclb{PlotErrInterpTcheb}\\
    de sorte que l'interpolation se fasse au niveau des abscisses de Tchebychev.

    \item Étudier comme dans l'exercice précédent la convergence des interpolations de Tchebychev pour la fonction $f_{3}$. Que constatez-vous ?
  \end{enumerate}
\end{exo}


%%%%%%%%%%%%%%%%%%%%%%%%%%%%%%%%%%%%%%%%%%%%%%%%%
\end{document}
