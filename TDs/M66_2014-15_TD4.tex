\documentclass[a4paper,12pt,reqno]{amsart}
\usepackage{macros_M66}
\newgeometry{top=21mm,left=23mm,right=21mm,bottom=21mm, nohead, nofoot}

\renewcommand{\theequation}{\fnsymbol{equation}}

\begin{document}
\hautdepage{TD4: EDO -- Différences Finies}


%-----------------------------------
\begin{exo}

  Soit le problème de Cauchy
    \[
      y''(x)+xy'(x)+(1+x)y(x)=x^2, \quad y(0)=0, \quad y'(0)=1.
    \]
  \begin{enumerate}
    \item Transformer cette EDO en un système différentiel du premier ordre équivalent.
    \item Effectuer deux étapes du schéma d'Euler explicite avec un pas $h=1/2$. Déterminer les approximations de $y$, $y'$ et $y''$ aux points $x_1=1/2$ et $x_2=1$.
  \end{enumerate}
\end{exo}

%-----------------------------------
\begin{exo}

  Soit l'EDO
    \[
      y'(x)=\sqrt{y(x)}, \quad y(0)=0.
    \]
  \begin{enumerate}
    \item Trouver une solution à cette EDO autre que la solution triviale $y \equiv 0$.
    \item Le théorème de Cauchy-Lipschitz assure l'unicité d'une solution. Quelle hypothèse du théorème n'est pas satisfaite ici?
    \item Que donne le schéma d'Euler explicite?
    \item Que donne le schéma d'Euler implicite?
    \item Montrer que pour une condition initiale $y(0) > 0$ la solution est unique. Décrire les solutions maximales dans ce cas.
    \item Comment peut-on essayer d'approcher ces solutions maximales avec un schéma d'Euler ? Que risque de se passer pour $t_{n} \rightarrow -\infty$ ?

  \end{enumerate}
\end{exo}

%-----------------------------------
\begin{exo}

  On considère le problème de Cauchy suivant:
    \begin{equation}\label{eq:sin}
      u'=\sin^3 u, \quad u(0)=u_{0}
    \end{equation}
  \begin{enumerate}
    \item Déterminer les solutions constantes du problème.
    \item Montrer que la fonction $f:u\mapsto \sin^3u$ est globalement lipschitzienne.
    \item Montrer que pour toute condition initiale le problèmes de Cauchy admet une unique solution. Décrire le comportement des solutions en fonction de $u_{0}$.
    \item Montrer que les solutions maximales sont globales (définies sur $\mathbb{R}$ tout entier).
    \item Soit $v$ et $w$ deux solutions de \eqref{eq:sin} pour deux conditions initiales différentes $v_{0}$ et $w_{0}$. Majorer la différence $|v(t)-w(t)|$ pour $t\geq 0$ par le lemme de Grönwall en fonction de $t$ et $|v_{0}-w_{0}|$. Est-ce une bonne estimation dans notre cas ?
    \item Écrire le schéma d'Euler explicite associé à une discrétisation uniforme de pas $h > 0$.
    \item Soit $u$ la solution exacte de \eqref{eq:sin} avec $u_{0} = \frac{\pi}{2}$. Soient $h \leq \frac{2}{3}$ et $u_{n}$ la $n$-ème valeur obtenue par le schéma d'Euler explicite. Montrer que $\lim_{n \rightarrow \infty}|u(nh)-u_{n}|=0$.

  \end{enumerate}
\end{exo}

%-----------------------------------
\begin{exo}

  On s'intéresse à la résolution numérique de l'équation différentielle ordinaire
  \[
    \left \{
      \begin{aligned}
        x'(t)  &= f(t,x(t)), \quad t\in I_0=[t_0,t_0+T],\\
        x(t_0) &= x_0.
      \end{aligned}
    \right .
  \]
  On suppose $f$ régulière. On introduit une subdivision uniforme $t_0<t_1<\cdots<t_N=t_0+T$ et on pose $h=t_{n+1}-t_n=T/N$, de telle sorte que $t_n=nh+t_0$ pour $0\leq n\leq N$.
  \begin{enumerate}
    \item Trouver $a$, $b$ et $c$ pour que la formule de quadrature suivante soit d'ordre maximal:
      \[
        \int_{t_n}^{t_{n+2}}\psi(t) dt + 4 \int_{t_n}^{t_{n+1}}\psi(t) dt
          \approx
            h \big( a\psi(t_{n+2})+b\psi(t_{n+1}) + c\psi(t_n)\big).
      \]
    \item Montrer que l'on peut approcher l'équation différentielle par le schéma numérique
      \[
        \left\{
          \begin{array}{l}
            X_0 \text{ et } X_1 \text{ donnés,}\\
            X_{n+2} + 4 X_{n+1} - 5 X_n = 6 h\Big(\dfrac{2}{3} f(t_{n+1},X_{n+1}) +
             \dfrac{1}{3}  f(t_n,X_n)\Big).
          \end{array}
        \right.
      \]
    \item On suppose désormais $f\equiv0$. Calculer $X_n$ en fonction de $X_0$ et $X_1$.
    \item On suppose, dans cette question uniquement, que $X_1 = X_0 = x_0$. Montrer que, pour tout $0 \leq n \leq N$, $e_n = |x(t_n)-X_n| = 0$.
    \item Que se passe-t-il lorsque $X_0 = x_0$ et $X_1 = X_0 + C h^p$?
  \end{enumerate}
\end{exo}

%-----------------------------------
\begin{exo}

  On considère l'équation différentielle ordinaire
    \[
      \left \{
        \begin{aligned}
          x'(t)+\lambda(t)x(t) &= b(t), \quad t\in I_0=\mathbb{R}_+,\\
          x(0) &= x_0>0.
        \end{aligned}
      \right .
    \]
  avec $\lambda$ et $b$ des fonctions régulières.
  \begin{enumerate}
    \item Écrire la solution de l'EDO.
    \item Montrer que si $b$ est une fonction positive, alors $x$ est positive pour toute fonction $\lambda$.
    \item Écrire le schéma d'Euler explicite associé à une discrétisation uniforme de pas $h>0$.
    \item On suppose désormais que $\lambda$ et $b$ sont des constantes, avec $b>0$ et $\lambda>0$. Sous quelle(s) condition(s) le schéma d'Euler conserve-t-il la positivité de la solution?
    \item Soient $\lambda=1$ et $b=2$. Représenter la solution. Faire de même avec la solution de l'approximation pour $x_0=1$ et $h=1/2$, $h=3/2$ et $h=5/2$, puis $x_0=5$ et $h=3/2$. Pouvait-on prévoir ces résultats?
  \end{enumerate}
\end{exo}

%-----------------------------------
\begin{exo}

  Soit $f$ une fonction de classe ${\mathcal{C}}^3$ sur $[0,T] \times \mathbb{R}$ à valeurs dans $\mathbb{R}$. On considère l'équation différentielle :
  \begin{equation}\label{equ1}
    \begin{cases}
      u'(t)=f(t,u(t)) \,, \quad t \in ]0,T] & \\
      u(0)=u_0 \,.
    \end{cases}
  \end{equation}
  Étant donné $K \in \mathbb{N}^*$, on fixe le pas de temps $\Delta t=T/K > 0$ et on note $t_n = n\, \Delta t$, $n = 0, \ldots, K$ et $U_n$  l'approximation de $u(t_n)$.

  On considère le schéma de Heun donné par la formule de récurrence pour $n=0,\dots, K-1$~:
  \[
    \begin{cases}
      V_{n+1} = U_n + \Delta t \, f(t_n,U_n) \,, \\
      U_{n+1} = U_n + \dfrac{\Delta t}{2} \, \big( f(t_n,U_n) + f(t_{n+1},V_{n+1}) \big) .
    \end{cases}
  \]
  \begin{enumerate}
    \item Interpréter graphiquement la méthode de Heun sur l'intervalle de temps $[t_n, t_{n+1}]$. Expliquer le dessin.

    \item Pour définir l'erreur de consistance de l'étape $n$ du schéma de Heun, on suppose : $U_n=u(t_n)$. Écrire $e_{n+1} = \varepsilon_{n+1} + \varepsilon_{n+1}^*$, où
    $\varepsilon_{n+1} = u(t_n+\Delta t) -u(t_n) - \Delta t \,u'(t_n + \dfrac{\Delta t}{2})$. Donner l'expression de $\varepsilon_{n+1}^*$ et montrer par un développement de Taylor que l'erreur $\varepsilon_{n+1}$ est en $\Delta t^3$, en précisant la constante.

    \item À l'aide de la formule de récurrence et d'un autre développement de Taylor, évaluer $\varepsilon_{n+1}^*$ en fonction de $\Delta t$ et montrer que le schéma de Heun est d'ordre $2$.

  \end{enumerate}

\end{exo}

%-----------------------------------
\begin{exo}

  Soit l'équation différentielle ordinaire suivante :
  \begin{equation}
    \label{equadif2}
    \left\{
      \begin{aligned}
        y'(t) &= y^2(t), \quad t \in [0,T], \quad T<1,\\
        y(0)  &= 1.
      \end{aligned}
    \right.
  \end{equation}
  \begin{enumerate}
    \item Donner la solution de l'équation différentielle (dont on supposera l'unicité).
    \item On choisit, pour la résolution de \eqref{equadif2}, le schéma d'Euler implicite à pas variable:
      \begin{equation}
        \label{EUI}
        y_{n+1} = y_n+ h_n \; f(t_{n+1},y_{n+1}).
      \end{equation}
      \begin{enumerate}
        \item Donner l'équation du second degré vérifiée par $y_{n+1}$ correspondant à l'utilisation du schéma \eqref{EUI} pour la résolution de \eqref{equadif2}.
        \item Donner la restriction sur le pas de temps $h_n$ à vérifier afin que cette équation admette deux racines réelles.
        \item Exprimer alors explicitement $y_{n+1}$ en fonction de $y_n$ et de $h_n$.
        \item Quel phénomène peut-on craindre si $T$ est trop proche de 1?
      \end{enumerate}
    \item Mêmes questions pour le schéma de Crank-Nicolson:
      \[
        y_{n+1}=y_n+ \frac{h_n}{2} \left[ f(t_n,y_n)+f(t_{n+1},y_{n+1}) \right].
      \]
  \end{enumerate}
\end{exo}

\end{document}
