\documentclass[a4paper,12pt,reqno]{amsart}
\usepackage{macros_M66}

\begin{document}
\hautdepage{TD1: Interpolation et splines}

% ==================================
\section{Interpolation}
% ==================================


%-------------------------------------
\begin{exo} (Différences divisées)

Soient $x_0,x_1,\ldots,x_n$ des points distincts d'un intervalle $I$, et $f$,$g$ et $h$ des applications réelles définies sur cette intervalle. On note par $f[x_{0},x_{1},\ldots,x_{n}]$ la différence divisée de $f$ en $x_0,x_1,\ldots,x_n$.
  \begin{enumerate}
    \item
      \begin{enumerate}
        \item Démontrer l'identité
          $$
            f[x_{0},x_{1},\ldots,x_{n}] =
              \sum_{j=0}^{n} f(x_{j}) \prod_{\substack{k=0 \\ k\neq j}}^{n}
                \frac{1}{x_{k} - x_{j}}\,.
          $$

        \item En déduire que la différence divisée $f[x_{0},x_{1},\ldots,x_{n}]$ est une fonction symétrique, c'est-à-dire, que pour toute permutation $\sigma$ de l'ensemble $\{0,\ldots,n\}$ dans lui même,
          $$
            f[x_{\sigma(0)},x_{\sigma(1)},\ldots,x_{\sigma(n)}] = f[x_{0},x_{1},\ldots,x_{n}]
          $$
      \end{enumerate}

    \item Si $f = \alpha g + \beta h$ pour certains $\alpha ,\beta \in \mathbb{R}$, alors
      $$
        f[x_{0},\ldots,x_{n}] = \alpha g[x_{0},\ldots,x_{n}] + \beta h[x_{0},\ldots,x_{n}]\,.
      $$

    \item Si $f = gh$, alors on a la formule suivante (appelée formule de Leibniz):
      $$
        f[x_{0},\ldots,x_{n}] = \sum_{j=0}^{n} g[x_{0},\ldots,x_{j}]h[x_{j},\ldots,x_{n}]\,.
      $$
  \end{enumerate}
\end{exo}


%-----------------------------------
\begin{exo} (Interpolation quadratique)
  \begin{enumerate}
    \item Soit $I$ un intervalle et $f\in C^3(I)$. On note $p_2(x)$ le polynôme de degré $\leq 2$ qui interpole la fonction $f$ aux points $x_i=x_0+ih \in I$ pour $i=0,1,2$. Montrer que
      $$
        \forall x\in [x_0, x_2] \quad
          | {f(x)-p_2(x)} | \leq \frac{h^3}{9\sqrt{3}}M,
      $$
    où $M$ est une constante ne dépendant que de la restriction de $f$ à $I$.

  \item On veut construire une table de valeurs de la fonction $f(x)=\sqrt{x+1}$ dans l'intervalle $[0,1]$ pour des points équidistants $x_{i+1}=x_i+h$.\newline
  Quelle valeur doit prendre $h$ pour garantir $7$ chiffres décimaux corrects en faisant une interpolation quadratique?
  \end{enumerate}
\end{exo}

%-------------------------------------
\begin{exo} (Interpolation d'Hermite)

  On se donne $n+1$ abscisses distinctes $x_0, x_1, \ldots , x_n$.
  On considère les polynômes $U_i(x)$ et $V_i(x)$ , $0\leq i\leq n$, de degré $2n+1$ qui vérifient les conditions suivantes:
  \begin{equation}\label{eq}
    \begin{array}{lcll}
      U_i(x_k)=\delta_{ik} & ; & U_i'(x_k)=0           & i,k=0,\ldots,n\\
      V_i(x_k)=0           & ; & V_i'(x_k)=\delta_{ik} & i,k=0,\ldots,n
    \end{array}
  \end{equation}
  \begin{enumerate}
    \item Quelles conditions d'interpolation vérifie le polynôme
      $$
        H(x)=\sum_{i=0}^n U_i(x)y_i +\sum_{i=0}^nV_i(x)y_i'\quad ?
      $$
    \item En sachant que les polynômes de la base de Lagrange associée aux nœuds $x_i$ vérifient $L_i(x_k)=\delta_{ik}$, montrer que
      \begin{eqnarray*}
        U_i(x) &=& \left[ 1-2L_i'(x_i)(x-x_i)\right](L_i(x))^2\\
        V_i(x) &=& (x-x_i)(L_i(x))^2
      \end{eqnarray*}
    vérifient les conditions (\ref{eq}).
    \item On considère le cas particulier de $n=1$, l'intervalle $I=[0,1]$ et soient $x_0=0$ et $x_1=1$.Soit $f(x)$ une fonction dans $C^4(I)$ et telle que
      $$
        f(x_i)=y_i, \qquad  f'(x_i)=y'_i, \qquad i=0,1.
      $$
    On définit la fonction $S(x)$, pour $x \in \mathring{I}$, par la relation
      $$
        f(x)=H(x)+x^2(x-1)^2S(x).
      $$
    \begin{enumerate}
      \item Soit $x$ fixé dans $\mathring{I}$. On introduit la fonction $F$ définie par
        $$
          F(t)=f(t)-H(t)-t^2(t-1)^2S(x)
        $$
      \begin{enumerate}
        \item Montrer que $F(t)$ s'annule en (au moins) 3 points distincts que l'on explicitera.
        \item Montrer qu'il existe 2 réels distincts $t^1_1$ et $t^1_2$ tels que
          $$
            F'(t^1_1)=F'(t^1_2)=0.
          $$
      \end{enumerate}
      \item Calculer $F'(t)$, $F'(0)$, $F'(1)$.\\
      En déduire qu'il existe (au moins) 4 réels distincts $t^2_i$, $i=1,2,3,4$ tels que
        $$
          F'(t^2_i)=0, \forall  \ i=1, \cdots , 4.
        $$
      \item En déduire alors qu'il existe un réel $\xi_x \in \mathring{I}$ tel que
        $$
          S(x)=\frac{f^{(4)}(\xi_x)}{4!}.
        $$
      \item Donner alors une majoration de l'erreur d'interpolation $|f(x)-H(x)|$.\\
    \end{enumerate}

    \item \textit{Application:} Une déviation entre deux voies de chemin de fer parallèles doit être une fonction $f \in C^{4}\left([0,4]\right)$ qui unit les positions $(0,0)$ et $(4,2)$ et est tangent dans ces points, aux droites $y=0$ et $y=2$ respectivement. Déterminer le polynôme $H$ de degré $\leq 3$ qui satisfait les mêmes conditions. Utiliser la question précédente pour majorer l'erreur $|f(x)-H(x)|$ dans l'intervalle $[1,2]$.

  \end{enumerate}
\end{exo}

% ==================================
\section{Méthode des moindres carrés}
% ==================================


%-------------------------------------
\begin{exo} (Droite de régression)

  Considérons les données $\{(x_{i} ,y_{i}) \,|\, i = 0,\ldots,n\}$ où $y_{i}$ peut être vue comme la valeur $f(x_{i})$ prise par une fonction $f$ au nœud $x_{i}$ .

  Soit
    $$
      \Phi(b_{0} ,b_{1}) =
        \sum_{i=0}^{n} \big[y_{i} - (b_{0} + b_{1} x_{i})\big]^{2}.
    $$

  \begin{enumerate}
    \item Déterminer $b_{0}$ et $b_{1}$ qui minimise $\Phi$.

    La droite ainsi déterminée $x \mapsto b_{0} + b_{1} x$, est dite \emph{droite des moindres carrés}, ou \emph{de régression linéaire}. Elle est la solution de degré $1$ du problème des moindres carrées pour les données $\{(x_{i} ,y_{i})\}$.

    \item Exprimer les coefficients $b_{0}$ et $b_{1}$ en fonction de la moyenne $M = \frac1{(n+1)} \sum_{i=0}^{n} x_{i}$  et de la variance $V = \frac1{(n+1)} \sum_{i=0}^{n} (x_{i}-M)^{2}$.
    \item  Vérifier que la droite de régression linéaire passe par le point dont l'abscisse est la moyenne des $\{x_{i}\}$ et l'ordonnée est la moyenne des $\{y_{i}\}$.
  \end{enumerate}


\end{exo}

%-------------------------------------
\begin{exo} (Le cas général)

  Étant donnée $\{(x_{i} ,y_{i}) \,|\, i = 0,\ldots,n\}$, une solution du problème des moindres carrées de degré $m$ est la donnée d'un polynôme $P \in \mathbb{P}_{m}[\mathbb{R}]$ qui minimise la quantité:
    $$
        \sum_{i=0}^{n} \big[y_{i} - P(x_{i})\big]^{2}.
    $$

  \begin{enumerate}
    \item Montrer que pour $m=n$ il y a une solution unique et déterminer la.
    \item Que peut on dire de l'ensemble des solutions dans le cas $m>n$ ?
    \item On munit $\mathbb{P}_{m}[\mathbb{R}]$ de la base canonique $1,X,\ldots,X^{m}$. Soit $(a_{0},\ldots,a_{n})$ les coordonnées de $P$ dans cette base et $\overline{P}=(a_{0},\ldots,a_{n}) \in \mathbb{R}^{n}$ ça représentation dans $\mathbb{R}^{n}$. Soit la matrice $V_{m}=\left[x_{i}^{k}\right]_{\substack{i=0,\ldots,n\\ k=0,\ldots,m}}$ et le vecteur $Y=(y_{0},\ldots,y_{n})$. En considérant $\overline{P}$ et $Y$ comme des vecteurs colonnes, montrer que
      $$
        \sum_{i=0}^{n} \big[y_{i} - P(x^{i})\big]^{2} = \big\| V_{m}\overline{P}-Y \big\|_{2}^{2}.
      $$
    \item Montrer que pour $m \leq n$ la solution du problème des moindres carrées existe et est unique.
    \item Justifié que $V_{n}$ est inversible, puis montrer que les coefficients du polynôme de Lagrange sont les coefficients du vecteur $V_{n}^{-1}Y$.
    \item Redémontrer la question b) en utilisant la question c).
  \end{enumerate}
\end{exo}


% ==================================
\section{Splines}
% ==================================

%-------------------------------------
\begin{exo} (Base de splines)

  Soient $a=x_0 < x_1 < \cdots < x_{n}=b$ des points de l'intervalle $[a,b]$ et $\mathcal{S}_{k}[a,b]$ l'ensemble des splines de degré $k$ relativement à ces $(n+1)$ points. Vérifier que tout $s_{k} \in \mathcal{S}_{k}[a,b]$ admet une écriture de la forme
  $$
    s_{k}(x) = \sum_{i=0}^{k} b_{i}x^{i} + \sum_{i=1}^{n-1} c_{i}(x - x_{i})_{+}^{k}\,,
  $$
  où $(t)_{+}=\frac{t+|t|}{2}$ et $(t)_{+}^{k}=\left[(t)_{+}\right]^{k}$. Puis conclure que
    $$
      \{1,x,x^{2},\ldots,x^{k},(x-x_{1})_{+}^{k},\ldots,(x-x_{n-1})_{+}^{k}\}
    $$
  est une base de $\mathcal{S}_{k}[a,b]$.
\end{exo}


%-------------------------------------
\begin{exo} (Splines normalisées)

  Soit $a=x_{0}<x_{2}<\cdots<x_{n}=b$ les $n+1$ nœuds des splines cubiques $\mathcal{S}_{3}(Y)$ qui interpolent les valeurs $Y=(y_{0},y_{1},\ldots,y_{n})$.
  \begin{enumerate}
    \item Montrer qu'il existe une unique spline $s \in \mathcal{S}_{3}(Y)$, dit \emph{naturelle}, telle que $s''(a)=s''(b)=0$.

    \item Étant donnés deux nombre $d_{a}$ et $d_{b}$, montrer qu'il existe une unique spline $s \in \mathcal{S}_{3}(Y)$, dit \emph{serrée} ou \emph{tendue}, telle que $s'(a)=d_{a}$ et $s'(b)=d_{b}$.

    \item Si $y_{0}=y_{n}$, montrer qu'il existe une unique spline $s \in \mathcal{S}_{3}(Y)$, dit \emph{cyclique} ou \emph{périodique}, telle que $s^{(k)}(a)=s^{(k)}(b)$ pour $k=0,1,2$.
  \end{enumerate}
\end{exo}


%-------------------------------------
\begin{exo} (Propriété de la norme minimale)
  \begin{enumerate}
    \item Soit $f \in C^{2}([a,b])$, et soit $s_{3}$ une spline cubique \textit{naturelle} qui interpole $f$ aux noeuds $\Delta=\{a=x_{0}<x_{2}<\cdots<x_{n}=b\}$. Montrer que
      $$
        \int_{a}^{b} [s_{3}''(x)]^{2} dx \leq \int_{a}^{b} [f''(x)]^{2} dx
      $$
    avec égalité si et seulement si $f=s_{3}$.\newline
    \textit{Indication:} Commencer par la formule
    $$
      \int_{a}^{b} [f''(x)-s_{3}''(x)]s_{3}''(x) dx
        = \sum_{1}^{n}\int_{x_{i-1}}^{x_{i}} [f''(x)-s_{3}''(x)]s_{3}''(x) dx
    $$

    \item Montrer que la propriété précédente est vrai également pour les splines \emph{serrées} qui satisfont $s_{3}'(a)=f'(a)$ et $s_{3}'(b)=f'(b)$.

    \item Soit $\Delta^{*}$ un sous-ensemble de l'ensemble $\Delta$ des nœuds, et $s_{3}^{*}$ la spline naturelle (resp. serrée) qui interpole $f$ en $\Delta^{*}$. Montrer que
      $$
        \int_{a}^{b} [(s_{3}^{*})''(x)]^{2} dx \leq \int_{a}^{b} [s_{3}''(x)]^{2} dx
      $$
    avec égalité si et seulement si $s_{3}$ est un polynôme au voisinage des nœuds manquant à $\Delta^{*}$.

  \end{enumerate}
\end{exo}



%-------------------------------------
\begin{exo} (Splines quartique)

  Soit $f \in C^{4}([a,b])$, et soit $s_{4}$ une spline de degré $4$ qui interpole $f$ aux nœuds $\{a=x_{0}<x_{2}<\cdots<x_{n}=b\}$. Montrer que
      $$
        ||f''-s_{4}'' ||_{L^{2}[a,b]}^{2} \leq \int_{a}^{b} [f(x)-s_{4}(x)]f^{4}(x) dx
      $$
    si l'une des conditions suivante est vérifiée:
    \begin{enumerate}
      \item $f'(a)=s_{4}'(a)$ et $f'(b)=s_{4}'(b)$;
      \item $f''(a)=s_{4}''(a)$ et $f''(b)=s_{4}''(b)$;
      \item $f^{(i)}$ et $s_{4}^{(i)}$ sont périodiques pour $i \leq 2$.
    \end{enumerate}
\end{exo}

%-------------------------------------
\begin{exo} (Splines cubiques à peu de nœuds)

  Soit $a=x_{0}<x_{2}<\cdots<x_{n}=b$ les $n+1$ nœuds des splines cubiques $\mathcal{S}_{3}$. Soit $s \in \mathcal{S}_{3}$ qui satisfait les conditions aux bords $s^{(k)}(a)=s^{(k)}(b)=0$ pour $k=0,1,2$.
  \begin{enumerate}
    \item Montrer que si $n < 4$ alors $s=0$.

    \item Montrer que si $n=4$ alors $s$ est uniquement déterminée par sa valeur en $x_{2}$.

    \item Calculer explicitement $s$ dans le cas des nœuds $\{-2,-1,0,1,2\}$ en fonction de sa valeur $c$ en $0$.
  \end{enumerate}

\end{exo}

%-------------------------------------
\begin{exo} (Base cardinale des splines cubiques)

  Soit $a=x_{0}<x_{2}<\cdots<x_{n}=b$ les $n+1$ nœuds des splines cubiques $\mathcal{S}_{3}$.

  Pour $i=0,\ldots,n$ on note par $\varphi_{i}$ la spline naturelle telle que $\varphi_{i}(x_{j})=\delta_{ij}$. Ainsi que les deux splines tendues $\varphi_{k}$ pour $k \in \{n+1,n+2\}$ définies par $\varphi_{k}(x_{i})=0$ pour $\forall i=0,\ldots,n$, $\varphi_{n+1}'(x_{0})=1$, $\varphi_{n+1}'(x_{n})=0$, $\varphi_{n+2}'(x_{0})=0$ et $\varphi_{n+2}'(x_{n})=1$.
  \begin{enumerate}
    \item Montrer que les $\{\varphi_{k}\}_{k=0,\ldots,n+2}$ forment une base de $\mathcal{S}_{3}$.\newline
    Cette base est appeler \emph{base cardinale}.
    \item Soit $x_{i}=i$ pour $i=0,1,2$. Calculer la base cardinale dans ce cas.
  \end{enumerate}
\end{exo}

%-------------------------------------
\begin{exo} (B-splines)

  Soit $a=x_{0}<x_{2}<\cdots<x_{n}=b$ les $n+1$ nœuds des splines $\mathcal{S}_{k}$ de degré $k$. Soit les fonctions $B_{i,j}$ définies par récurrence comme suit
    $$
      \left\{
      \begin{array}{rll}
        B_{i,0}(x) & =
          \Bigg\{
          \begin{array}{rl}
            1 & \text{ si } x \in [x_{i},x_{i+1}]\\
            0 & \text{ sinon }
          \end{array}               & \text{ pour } i=0,\ldots,n-1\\[3ex]
        B_{i,k}(x) & = \frac{x-x_{i}}{x_{i+k}-x_{i}}B_{i,k-1}(x)
                  + \frac{x_{i+k+1}-x}{x_{i+k+1}-x_{i+1}}B_{i+1,k-1}(x)
                                    & \text{ pour }
                                      \left\{
                                        \begin{array}{r}
                                          \forall\ k \geq 1, i \geq 0\\
                                           i+k+1 \leq n
                                        \end{array}
                                      \right.
      \end{array}
      \right.
    $$\vspace{0pt}
  \begin{enumerate}
    \item Calculer et tracer les $B_{i,1}$ pour $i=0,\ldots,n-2$.
    \item Montrer que la restriction des $B_{i,1}$ à $I_{1}=[x_{1},x_{n-1}]$ est une base de $1$-splines respectivement aux nœuds $\{x_{1},\ldots,x_{n-1}\}$.
    \item Montrer que le support de $B_{i,k}$ est contenu dans $[x_{i},x_{i+k+1}]$.
    \item Montrer que
      $$
        B_{i,k}(x) = (x_{i+k+1}-x_{i})
            \sum_{j=0}^{k+1}
              \frac{(x_{i+j}-x)_{+}^{k}}{\displaystyle\prod_{\substack{m=0 \\ m\neq j}}^{m=k+1}(x_{i+j}-x_{i+m})}
      $$
    \item En déduire que les $B_{i,k}$ sont des splines de degré $k$ linéairement indépendant.\newline
      On appelle les $B_{i,k}$ les \emph{B-splines} de degré $k$.
    \item Soit $I_{k}=[x_{k},x_{n-k}]$ et  $\mathcal{S}_{k}[I_{k}]$ les splines de degré $k$ sur $I_{k}$ ayant pour nœuds $\{x_{k},\ldots,x_{n-k}\}$. Montrer $\{\left.B_{i,k}\right|_{I_{k}} \,|\, i=-k+1,\ldots,n-1 \}$ forme une base de $\mathcal{S}_{k}[I_{k}]$.
    \item Comment peut on construire une base de $\mathcal{S}_{k}$ avec des B-splines ?
  \end{enumerate}
\end{exo}

\end{document}
